\chapter{stopCleaner}

\texttt{stopCleaner} can remove stop characters at the end of a sequence and furthermore allows to remove pseudogenes (as defined by having a stop codon in the middle of the sequence) from a set. 

\begin{table}[H]
\caption{General options for \textit{seqExtract}}
\begin{tabular}{lcp{12cm}}\hline
option & default & description\\
\hline
-h, --help & - & Produces a simple help message\\
  -i, --in &-&              The input file: Protein sequences in fasta format. If none is provided sequences are read from stdin.\\
  -o, --out &-&    The output file. If none is provided sequences will be printed to stdout\\
  --no-final-stop-removal  &-&     Do not remove the final stop characters.\\
  -r, --remove-pseudogenes &-&   Remove pseudogenes. Pseudogenes are in this case all genes that do not contain a stop at the end.\\
  -s, --stop   &.*   &      The stop characters to use.\\
  -l, --list  &-&            Produces a list of genes that were removed and writes it to the provided file.\\
  -k, --keep  &-&            Keep sequences with these IDs. Can be useful to prevent removal of sequences that are known to correctly contain a stop codon (etc. selenoproteins)\\
  -R, --replace &-&          Replace stops with this chararcter.\\
  \hline
\end{tabular}
\end{table}


Examples:

\begin{lstlisting}
./stopCleaner -i dmel.fa
\end{lstlisting}