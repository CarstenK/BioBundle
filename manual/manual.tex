\documentclass[12pt,a4paper,oneside]{scrreprt}

\usepackage{float}
\usepackage{listings}
\usepackage{color}
\setlength{\parindent}{0pt}


\definecolor{mygreen}{rgb}{0,0.6,0}
\definecolor{mygray}{rgb}{0.5,0.5,0.5}
\definecolor{mymauve}{rgb}{0.58,0,0.82}

\lstset{ %
  backgroundcolor=\color{white},   % choose the background color; you must add \usepackage{color} or \usepackage{xcolor}
  basicstyle=\footnotesize\ttfamily,        % the size of the fonts that are used for the code
  breakatwhitespace=false,         % sets if automatic breaks should only happen at whitespace
  breaklines=true,                 % sets automatic line breaking
  captionpos=b,                    % sets the caption-position to bottom
  commentstyle=\color{mygreen},    % comment style
  deletekeywords={...},            % if you want to delete keywords from the given language
  escapeinside={\%*}{*)},          % if you want to add LaTeX within your code
  extendedchars=true,              % lets you use non-ASCII characters; for 8-bits encodings only, does not work with UTF-8
  frame=single,	                   % adds a frame around the code
  keepspaces=true,                 % keeps spaces in text, useful for keeping indentation of code (possibly needs columns=flexible)
  keywordstyle=\color{blue},       % keyword style
  language=bash,                 % the language of the code
  otherkeywords={*,...},           % if you want to add more keywords to the set
  numbers=none,                    % where to put the line-numbers; possible values are (none, left, right)
  numbersep=5pt,                   % how far the line-numbers are from the code
  numberstyle=\tiny\color{mygray}, % the style that is used for the line-numbers
  rulecolor=\color{black},         % if not set, the frame-color may be changed on line-breaks within not-black text (e.g. comments (green here))
  showspaces=false,                % show spaces everywhere adding particular underscores; it overrides 'showstringspaces'
  showstringspaces=false,          % underline spaces within strings only
  showtabs=false,                  % show tabs within strings adding particular underscores
  stepnumber=2,                    % the step between two line-numbers. If it's 1, each line will be numbered
  stringstyle=\color{mymauve},     % string literal style
  tabsize=2,	                   % sets default tabsize to 2 spaces
  title=\lstname                   % show the filename of files included with \lstinputlisting; also try caption instead of title
}


\title{BioBundle v1.1- User Manual}
\author{Carsten Kemena}

\begin{document}
\maketitle

\tableofcontents

\chapter{seqExtract}

\texttt{seqExtract} is a small program to extract sequences from a sequence set. It allows to extract sequences many different criteria.


\section{Options}
General options:

\begin{table}[H]
\caption{General options for \textit{seqExtract}}
\begin{tabular}{lp{12cm}}\hline
option & description\\
\hline
  -h --help & Displays a simple help message\\
  -i --in & The input file. Currently only the \texttt{fasta}-format is supported.\\
  -I --index &  Uses an index file. This file can speed up sequence extraction in larger files (e.g. genomes) especially when rerunning the command several times. One can specify the path to the index file using the 'indexFile' parameter. If none is specified a file will be created with same name as input file but with an additional '.sei' extension. \\
  -F --indexFile & The path to the index file if not the default position should be used.\\
  -l --inputList & A file containing a list of files to be used as input\\
  \hline
\end{tabular}
\end{table}






Output options:
\begin{table}[H]
\caption{Output options for \textit{seqExtract}}
\begin{tabular}{lp{12cm}}
\hline
Option & Description\\
\hline
  -o, --out & The output file. Currently only \texttt{fasta}-format output is supported.\\
  -a, --append &  Appends the extracted sequences to a file. (default: false)\\
  -c, --remove-comments &   Remove comments from the fasta headers. (default: false)\\
  \hline
\end{tabular}
\end{table}


Extract options:
\begin{table}[H]
\caption{Extract options for \textit{seqExtract}}
\begin{tabular}{llp{12cm}}
\hline
Option & Default value & Description\\
\hline
  -e, --extract        & The sequence to extract\\
  -d, --delimiter      & The delimiter to use\\
  -E, --regex          & A regular expression to match a name\\
  -r, --remove         & Remove the given sequences\\
  -n, --numSeqs        & The number of sequences to extract\\
  -s, --seed           & Seed for random extract function\\
  -f, --function       & The function to use for extraction\\
  -m, --ignore-missing & Ignore missing sequences\\
  \hline
\end{tabular}
\end{table}



\begin{table}[H]
\caption{Modifying options for \textit{seqExtract}}
\begin{tabular}{llp{12cm}}
\hline
Option & Default value & Description\\
\hline
  -t, --translate &                 Translate into amino acid\\
  -T, --table & arg (=standard)     The translation table to use\\
  -R, --revComp &                   Calculate the reverse complement\\
  \hline
\end{tabular}
\end{table}
\chapter{isoformCleaner}

\texttt{isoformCleaner} is a small program to remove isoforms from a set and keep only the largest one. This is often useful as for many analyses isoforms will influence the results.   


\section{Options}


This section explains all the parameters that can be used with the program.
\subsection{General options}

These are the general options that can be used.
\begin{table}[H]
\caption{General options for \textit{seqExtract}}
\begin{tabular}{lcp{10cm}}\hline
option & default & description\\
\hline
option 	&default 	&effect\\
-h, --help &	- 	&Prints a simple help message\\
-i, --in &	- 	&The sequence file to clean\\
-o, --out &	- 	&The output file\\
-s, --splitchar& 	-& 	The split character to use to distinguish gene name from isoform extension (e.g. “-” in “Gene1-PA”).\\
\end{tabular}
\end{table}

\subsection{Regex options}

These are the options that can be used for a regular expression based cleaning. The regular expression should identify a section of the sequence header that is the same for all isoforms of a gene (e.g. the gene name).

\begin{table}[H]
\caption{General options for \textit{seqExtract}}
\begin{tabular}{lcp{10cm}}\hline
option & default & description\\
\hline
option &	default &	effect\\
-r, --regular 	&- 	&Regular expression\\
-c, --comment 	&- 	&Search comment only\\
-n, --name 	&- 	&Search name only\\
-p, --preset &	 -  	&Preset regex.\\\hline
\end{tabular}
\end{table}

For two common patterns some regular expressions have precoded. The availabele presets can be seend in the table below.
\begin{center}
 \begin{table}[H]
\caption{Preset options}
\begin{tabular}{ll}\hline
name &regular expression\\
gene & \verb| gene[:=]\s*([\S]+) |\\
flybase& \verb| parent=(FBgn[^ ,]+,) |\\\hline
\end{tabular}
\end{table}
\end{center}

\subsection{GFF options}

If none of the above works there is the chance that you can use a gff to do the isoform cleaning.
This is \textcolor{red}{highly experimental} and only works if the Parent field is present. Furthermore currently there is no support for multiple parents.

\begin{table}[H]
\caption{GFF options}
\begin{tabular}{lcp{10cm}}\hline
option & default & description\\
\hline
-g, --gff 	&-& 	The gff file in gff3 format.\\
-t, --type 	&mRNA& 	The feature type that contains the sequence name as ID and the gene name in the parent field.\\
-f, --id-field 	&ID& 	The argument is used to identify the field in the GFF-file that contains the sequence names. Usually ID is correct.\\
\end{tabular}
\end{table}


\section{Examples}

Here are some very basic example on the usage of isoform cleaner.

\begin{lstlisting}
# use a split character
isoformCleaner -s '.' -i foo.fa -o bar.fa

# usa a regular expression
isoformCleaner -r "parent=(FBgn[^ ,]+,)" -i foo.fa -o bar.fa

# same as above but with a preset value
isoformCleaner -p flybase -i foo.fa -o bar.fa

# gff cleaning
isoformCleaner -i foo.fa -g bar.gff
\end{lstlisting}

\chapter{stopCleaner}

\texttt{stopCleaner} can remove stop characters at the end of a sequence and furthermore allows to remove pseudogenes (as defined by having a stop codon in the middle of the sequence) from a set. 

\begin{table}[H]
\caption{General options for \textit{seqExtract}}
\begin{tabular}{lcp{12cm}}\hline
option & default & description\\
\hline
-h, --help & - & Produces a simple help message\\
  -i, --in &-&              The input file: Protein sequences in fasta format. If none is provided sequences are read from stdin.\\
  -o, --out &-&    The output file. If none is provided sequences will be printed to stdout\\
  --no-final-stop-removal  &-&     Do not remove the final stop characters.\\
  -r, --remove-pseudogenes &-&   Remove pseudogenes. Pseudogenes are in this case all genes that do not contain a stop at the end.\\
  -s, --stop   &.*   &      The stop characters to use.\\
  -l, --list  &-&            Produces a list of genes that were removed and writes it to the provided file.\\
  -k, --keep  &-&            Keep sequences with these IDs. Can be useful to prevent removal of sequences that are known to correctly contain a stop codon (etc. selenoproteins)\\
  -R, --replace &-&          Replace stops with this chararcter.\\
  \hline
\end{tabular}
\end{table}


Examples:

\begin{lstlisting}
./stopCleaner -i dmel.fa
\end{lstlisting}


\end{document}